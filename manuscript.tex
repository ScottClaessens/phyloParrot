% Options for packages loaded elsewhere
\PassOptionsToPackage{unicode}{hyperref}
\PassOptionsToPackage{hyphens}{url}
%
\documentclass[
  man,floatsintext]{apa6}
\usepackage{amsmath,amssymb}
\usepackage{iftex}
\ifPDFTeX
  \usepackage[T1]{fontenc}
  \usepackage[utf8]{inputenc}
  \usepackage{textcomp} % provide euro and other symbols
\else % if luatex or xetex
  \usepackage{unicode-math} % this also loads fontspec
  \defaultfontfeatures{Scale=MatchLowercase}
  \defaultfontfeatures[\rmfamily]{Ligatures=TeX,Scale=1}
\fi
\usepackage{lmodern}
\ifPDFTeX\else
  % xetex/luatex font selection
\fi
% Use upquote if available, for straight quotes in verbatim environments
\IfFileExists{upquote.sty}{\usepackage{upquote}}{}
\IfFileExists{microtype.sty}{% use microtype if available
  \usepackage[]{microtype}
  \UseMicrotypeSet[protrusion]{basicmath} % disable protrusion for tt fonts
}{}
\makeatletter
\@ifundefined{KOMAClassName}{% if non-KOMA class
  \IfFileExists{parskip.sty}{%
    \usepackage{parskip}
  }{% else
    \setlength{\parindent}{0pt}
    \setlength{\parskip}{6pt plus 2pt minus 1pt}}
}{% if KOMA class
  \KOMAoptions{parskip=half}}
\makeatother
\usepackage{xcolor}
\usepackage{graphicx}
\makeatletter
\def\maxwidth{\ifdim\Gin@nat@width>\linewidth\linewidth\else\Gin@nat@width\fi}
\def\maxheight{\ifdim\Gin@nat@height>\textheight\textheight\else\Gin@nat@height\fi}
\makeatother
% Scale images if necessary, so that they will not overflow the page
% margins by default, and it is still possible to overwrite the defaults
% using explicit options in \includegraphics[width, height, ...]{}
\setkeys{Gin}{width=\maxwidth,height=\maxheight,keepaspectratio}
% Set default figure placement to htbp
\makeatletter
\def\fps@figure{htbp}
\makeatother
\setlength{\emergencystretch}{3em} % prevent overfull lines
\providecommand{\tightlist}{%
  \setlength{\itemsep}{0pt}\setlength{\parskip}{0pt}}
\setcounter{secnumdepth}{-\maxdimen} % remove section numbering
% Make \paragraph and \subparagraph free-standing
\ifx\paragraph\undefined\else
  \let\oldparagraph\paragraph
  \renewcommand{\paragraph}[1]{\oldparagraph{#1}\mbox{}}
\fi
\ifx\subparagraph\undefined\else
  \let\oldsubparagraph\subparagraph
  \renewcommand{\subparagraph}[1]{\oldsubparagraph{#1}\mbox{}}
\fi
\newlength{\cslhangindent}
\setlength{\cslhangindent}{1.5em}
\newlength{\csllabelwidth}
\setlength{\csllabelwidth}{3em}
\newlength{\cslentryspacingunit} % times entry-spacing
\setlength{\cslentryspacingunit}{\parskip}
\newenvironment{CSLReferences}[2] % #1 hanging-ident, #2 entry spacing
 {% don't indent paragraphs
  \setlength{\parindent}{0pt}
  % turn on hanging indent if param 1 is 1
  \ifodd #1
  \let\oldpar\par
  \def\par{\hangindent=\cslhangindent\oldpar}
  \fi
  % set entry spacing
  \setlength{\parskip}{#2\cslentryspacingunit}
 }%
 {}
\usepackage{calc}
\newcommand{\CSLBlock}[1]{#1\hfill\break}
\newcommand{\CSLLeftMargin}[1]{\parbox[t]{\csllabelwidth}{#1}}
\newcommand{\CSLRightInline}[1]{\parbox[t]{\linewidth - \csllabelwidth}{#1}\break}
\newcommand{\CSLIndent}[1]{\hspace{\cslhangindent}#1}
\ifLuaTeX
\usepackage[bidi=basic]{babel}
\else
\usepackage[bidi=default]{babel}
\fi
\babelprovide[main,import]{english}
% get rid of language-specific shorthands (see #6817):
\let\LanguageShortHands\languageshorthands
\def\languageshorthands#1{}
% Manuscript styling
\usepackage{upgreek}
\captionsetup{font=singlespacing,justification=justified}

% Table formatting
\usepackage{longtable}
\usepackage{lscape}
% \usepackage[counterclockwise]{rotating}   % Landscape page setup for large tables
\usepackage{multirow}		% Table styling
\usepackage{tabularx}		% Control Column width
\usepackage[flushleft]{threeparttable}	% Allows for three part tables with a specified notes section
\usepackage{threeparttablex}            % Lets threeparttable work with longtable

% Create new environments so endfloat can handle them
% \newenvironment{ltable}
%   {\begin{landscape}\centering\begin{threeparttable}}
%   {\end{threeparttable}\end{landscape}}
\newenvironment{lltable}{\begin{landscape}\centering\begin{ThreePartTable}}{\end{ThreePartTable}\end{landscape}}

% Enables adjusting longtable caption width to table width
% Solution found at http://golatex.de/longtable-mit-caption-so-breit-wie-die-tabelle-t15767.html
\makeatletter
\newcommand\LastLTentrywidth{1em}
\newlength\longtablewidth
\setlength{\longtablewidth}{1in}
\newcommand{\getlongtablewidth}{\begingroup \ifcsname LT@\roman{LT@tables}\endcsname \global\longtablewidth=0pt \renewcommand{\LT@entry}[2]{\global\advance\longtablewidth by ##2\relax\gdef\LastLTentrywidth{##2}}\@nameuse{LT@\roman{LT@tables}} \fi \endgroup}

% \setlength{\parindent}{0.5in}
% \setlength{\parskip}{0pt plus 0pt minus 0pt}

% Overwrite redefinition of paragraph and subparagraph by the default LaTeX template
% See https://github.com/crsh/papaja/issues/292
\makeatletter
\renewcommand{\paragraph}{\@startsection{paragraph}{4}{\parindent}%
  {0\baselineskip \@plus 0.2ex \@minus 0.2ex}%
  {-1em}%
  {\normalfont\normalsize\bfseries\itshape\typesectitle}}

\renewcommand{\subparagraph}[1]{\@startsection{subparagraph}{5}{1em}%
  {0\baselineskip \@plus 0.2ex \@minus 0.2ex}%
  {-\z@\relax}%
  {\normalfont\normalsize\itshape\hspace{\parindent}{#1}\textit{\addperi}}{\relax}}
\makeatother

% \usepackage{etoolbox}
\makeatletter
\patchcmd{\HyOrg@maketitle}
  {\section{\normalfont\normalsize\abstractname}}
  {\section*{\normalfont\normalsize\abstractname}}
  {}{\typeout{Failed to patch abstract.}}
\patchcmd{\HyOrg@maketitle}
  {\section{\protect\normalfont{\@title}}}
  {\section*{\protect\normalfont{\@title}}}
  {}{\typeout{Failed to patch title.}}
\makeatother

\usepackage{xpatch}
\makeatletter
\xapptocmd\appendix
  {\xapptocmd\section
    {\addcontentsline{toc}{section}{\appendixname\ifoneappendix\else~\theappendix\fi\\: #1}}
    {}{\InnerPatchFailed}%
  }
{}{\PatchFailed}
\keywords{tool use; technical intelligence; parrots; crowdsourcing; phylogenetic modelling}
\usepackage{lineno}

\linenumbers
\usepackage{csquotes}
\usepackage{array}
\usepackage{caption}
\usepackage{graphicx}
\usepackage{siunitx}
\usepackage[normalem]{ulem}
\usepackage{colortbl}
\usepackage{multirow}
\usepackage{hhline}
\usepackage{calc}
\usepackage{tabularx}
\usepackage{threeparttable}
\usepackage{wrapfig}
\usepackage{adjustbox}
\usepackage{hyperref}
\usepackage{setspace}
\raggedbottom
\AtBeginEnvironment{tabular}{\singlespacing}
\AtBeginEnvironment{lltable}{\singlespacing}
\AtBeginEnvironment{tablenotes}{\doublespacing}
\captionsetup[table]{font={stretch=1,small}}
\captionsetup[figure]{font={stretch=1,small}}
\nolinenumbers
\note{\raggedright † These authors contributed equally to this work \par ‡ These authors contributed equally to this work \par * Correspondence concerning this article should be addressed to Scott Claessens, Level 2, Building 302, 23 Symonds Street, Auckland, New Zealand. \text{E-mail:} \href{scott.claessens@gmail.com}{\nolinkurl{scott.claessens@gmail.com}} \par This working paper has not yet been peer-reviewed.}
\ifLuaTeX
  \usepackage{selnolig}  % disable illegal ligatures
\fi
\IfFileExists{bookmark.sty}{\usepackage{bookmark}}{\usepackage{hyperref}}
\IfFileExists{xurl.sty}{\usepackage{xurl}}{} % add URL line breaks if available
\urlstyle{same}
\hypersetup{
  pdftitle={Crowdsourcing and phylogenetic modelling reveal parrot tool use is not rare},
  pdfauthor={Amalia P. M. Bastos†,1,2, Scott Claessens*†,2, Ximena J. Nelson3, David Welch4, Quentin D. Atkinson‡,2, \& Alex H. Taylor‡,2,3,5,6},
  pdflang={en-EN},
  pdfkeywords={tool use; technical intelligence; parrots; crowdsourcing; phylogenetic modelling},
  hidelinks,
  pdfcreator={LaTeX via pandoc}}

\title{Crowdsourcing and phylogenetic modelling reveal parrot tool use is not rare}
\author{Amalia P. M. Bastos\textsuperscript{†,1,2}, Scott Claessens\textsuperscript{*†,2}, Ximena J. Nelson\textsuperscript{3}, David Welch\textsuperscript{4}, Quentin D. Atkinson\textsuperscript{‡,2}, \& Alex H. Taylor\textsuperscript{‡,2,3,5,6}}
\date{}


\shorttitle{Parrot tool use}

\affiliation{\vspace{0.5cm}\textsuperscript{1} \footnotesize Department of Psychological \& Brain Sciences, Johns Hopkins University, Baltimore, MD, United States\\\textsuperscript{2} \footnotesize School of Psychology, University of Auckland, Auckland, New Zealand\\\textsuperscript{3} \footnotesize School of Biological Sciences, University of Canterbury, Christchurch, New Zealand\\\textsuperscript{4} \footnotesize School of Computer Science, University of Auckland, Auckland, New Zealand\\\textsuperscript{5} \footnotesize ICREA, Pg. Lluís Companys 23, Barcelona, Spain\\\textsuperscript{6} \footnotesize Institute of Neuroscience, Universitat Autònoma de Barcelona, Barcelona, Spain}

\abstract{%
Putatively rare behaviours, such as tool use, are challenging to study because absence of evidence can arise either from a species' inability to produce the behaviour or from insufficient research effort. Here, we tackle this challenge by combining crowdsourcing and phylogenetic modelling to estimate actual rates of tool use in parrots. Crowdsourcing on a social media platform revealed novel instances of tool use in 17 parrot species, more than doubling the confirmed number of tool-using parrot species from 11 (3\%) to 28 (7\%). Phylogenetic modelling ranked additional species that are most likely to be unobserved tool users, suggesting that between 11\% and 17\% of extant parrot species may be tool users. These discoveries have implications for inferences about the evolutionary drivers and origins of tool use in parrots, revealing associations with relative brain size and feeding generalism and indicating several genera where tool use was likely an ancestral trait. Overall, our findings challenge the assumption that current sampling effort captures the full distribution of putatively rare animal behaviours. Combining our sampling and analysis methods offers a fruitful approach for investigating the distribution, drivers, and origins of other rare behaviours.
}



\begin{document}
\maketitle

\linenumbers

Our understanding of the evolution of animal behaviour is built on the
assumption that we have access to sufficient data\textsuperscript{1--3}. However, this is not always the case. Data on behaviours that are
rare, fleeting, or otherwise difficult to observe are likely to be patchy and
incomplete\textsuperscript{4,5}. Among species for which such behaviours
have not been observed, it can be difficult to differentiate between cases in
which the species is truly incapable of producing the behaviour and cases in
which the species is capable of producing the behaviour but the behaviour has
not yet been observed. Such a distinction can be critical for drawing
conclusions about the rarity and evolution of the behaviour in question.

Comparative work on the evolution of tool use is a paradigmatic example of this
issue. The initial discoveries of tool use in chimpanzees\textsuperscript{6},
birds\textsuperscript{7}, dolphins\textsuperscript{8}, and octopuses\textsuperscript{9} occurred
decades after significant advances on other more easily measurable aspects of
their biology. Since then, scholars have proposed a clear operational
definition of tool use applicable to all species --- the manipulation of an
unattached object as an extension of the animal's body to achieve a
goal\textsuperscript{10} --- and have used the distribution of species meeting this
definition to make various claims about the evolutionary drivers of tool use
behaviours. For example, based on the observation that bird species with
reported tool use tended to have larger brains, researchers have identified
higher relative brain size as a likely precondition for tool using
capabilities\textsuperscript{11--13}. These researchers
argue that larger brains are better able to integrate visual and somatosensory
information when innovating novel behaviours, such as tool use, in changing
environments\textsuperscript{14,15}. Similarly, researchers have used existing
reports of tool use in birds to debate the roles of generalist versus
specialist feeding strategies in driving the evolution of tool use, with some
arguing that feeding generalists require technical innovations to expand their
dietary niche\textsuperscript{14,16,17} and others arguing that
feeding specialists require technical innovations for extractive foraging of
specific foods\textsuperscript{18,19}.

However, before we can make claims of this kind, we need to know whether current
research effort in the literature is sufficient for robust conclusions to be
drawn about the evolution of tool use. In fact, evidence suggests that research
effort is often systematically biased towards particular taxonomic groups, parts
of the world that are easy to access, and species with life history traits that
make them easier to study, such as larger distribution ranges and population
sizes\textsuperscript{20}. This is a crucial limitation because insufficient
observation may lead researchers to miss true instances of tool behaviours and
thus draw premature conclusions about the evolutionary drivers and origins of
tool use. While researchers have attempted to deal with this problem by
controlling for the number of scientific papers published on different species,
previous work has not yet attempted to quantify and explicitly model the
relationship between actual tool-using behaviour and what is reported in the
scientific literature. If more tool-using species exist than previously thought,
this could have important implications for theories of the evolutionary drivers
and origins of tool use and for our understanding of how rare this behaviour
actually is.

One potentially powerful method for quantifying actual rates of rare animal
behaviours is crowdsourcing\textsuperscript{21}. In a crowdsourcing study, researchers
collect reports from the general public and/or collate and analyse videos posted
on social media platforms. This citizen science approach has been widely used in
ecology to monitor the distributional patterns of species\textsuperscript{22}, but
has also recently been used to uncover a variety of rare animal behaviours,
including interspecies play in dogs\textsuperscript{21}, novel problem-solving
behaviours in horses\textsuperscript{23}, and socially-learned foraging innovations
in cockatoos\textsuperscript{24,25}. By casting the net wider than the
scientific literature, the crowdsourcing method can provide an indication of the
tool-using species that the literature might be missing.

Even after using this crowdsourcing approach, some tool-users could \emph{still}
remain unobserved. One principled framework for identifying these unobserved species
is to specify a causal model of the process that generates the observed data. We
propose one such causal model in Figure \ref{fig:plotDAG}. In this model, we
assume that the presence of tool use in the scientific literature (or in
crowdsourced reports) is caused by both unobserved tool use capabilities and the
number of published studies (or the number of crowdsourced reports) for any
given species. Tool users are more likely to be observed if they are well
studied, but understudied tool users may go undetected. Furthermore, based on
existing theories of the evolution of tool use\textsuperscript{11--19}, we propose that the
unobserved tool use capabilities are additionally caused by relative brain size,
feeding strategy, and shared phylogenetic ancestry. Expressing this causal model
as a statistical model can suggest further species which are likely to be
unobserved tool-users and, simultaneously, test existing theories of the
evolutionary drivers of tool use without incorrectly assuming that absence of
evidence is evidence of absence.












\begin{figure}
\centering
\includegraphics{manuscript_files/figure-latex/plotDAG-1.pdf}
\caption{\label{fig:plotDAG}\emph{Causal model of observed tool use.} Directed acyclic graph
of the causal relationships between observed tool use and other variables.
Available scientific data on tool use is caused both by unobserved tool use
presence and scientific research effort (i.e., number of publications).
Available crowdsourced data on tool use is caused both by unobserved tool use
presence and crowdsourcing effort (i.e., number of crowdsourced reports).
According to theory, unobserved tool use presence should be caused by relative
brain size (encephalisation quotient) and feeding strategy (generalist vs.~
specialist). These variables all share unobserved confounds generated by shared
phylogenetic history. Grey circles indicate unobserved variables.}
\end{figure}

Here, we apply these crowdsourcing and phylogenetic modelling approaches to tool
use in the parrot order. We focus on tool use in parrots for a number of
reasons. First, the scientific literature suggests that only a small proportion
of extant parrot species (11 out of 398; 3\%) use tools\textsuperscript{11,26--38}. Parrot tool use
thus provides an ideal test case for examining how robust sampling is in the
scientific literature. Second, parrots are highly popular as pets. Over 70\% of
all extant parrot species are bred in the aviculture industry and kept as pets
worldwide\textsuperscript{39--45}, enabling us to leverage the power of crowdsourcing on a social
media platform to search for evidence of tool use\textsuperscript{21}. Third, detailed
data on relative brain sizes\textsuperscript{46--51}, feeding strategies\textsuperscript{52}, and
shared ancestry\textsuperscript{53} exist for parrots, allowing us to fit the
statistical model implied by Figure \ref{fig:plotDAG} to the entire parrot
order.

We first present the results from our crowdsourcing survey, in which we collated
videos of tool use in parrots from an online video platform. This survey reveals
a number of previously unidentified tool-using parrot species, which we map onto
the phylogeny of the parrot order. We then describe our statistical model in
more detail, and use it to (\emph{i}) rank further parrot species that are likely
unobserved tool users and (\emph{ii}) re-examine key hypotheses regarding the
evolutionary drivers and origins of tool use in parrots.

\hypertarget{results}{%
\section{Results}\label{results}}

\hypertarget{crowdsourcing-reveals-tool-use-in-additional-parrot-species}{%
\subsection{Crowdsourcing reveals tool use in additional parrot species}\label{crowdsourcing-reveals-tool-use-in-additional-parrot-species}}

We surveyed the social media platform YouTube for video evidence of tool use in
parrots (see Methods for detailed search criteria). In our search, we used the
standard criteria for identifying tool use in the literature, defining ``true''
tool use behaviour as the manipulation of an unattached object as an extension
of the animal's body to achieve a goal\textsuperscript{10}, while ``borderline'' tool
use involved the use of an object that was still attached to a substrate\textsuperscript{54}.

In total, we found 116 videos of
104 individuals from
25
parrot species performing behaviours that met the definition of either true tool
use (100 videos of
89
individuals from
22
species) or borderline tool use (16 videos of
16
individuals from
7
species). All videos featured pet parrots in captive settings. In
68 of these
videos, owners did not appear to interact with the subjects. In
43 videos,
there was potential human interaction, either from the owners being in close physical
contact with the bird (e.g., bird perching on hand), talking to the bird, or
handing it the tool (which occurred in only two videos). We could not establish
the degree of human interaction in the remaining
5
videos, as sound had been removed or was substituted by music. None of the
videos featured owners directly rewarding tool use behaviours with food. All
borderline tool use cases were excluded from further analyses.

Of the 22
parrot species performing true tool use,
13
were represented in our video survey by two or more individuals over multiple
independent observations. True tool use always involved the subject using an
object for self-scratching (95
videos involved scratching the head and/or neck). The most common tool (53
videos) was a moulted feather. Human-made objects (e.g., pens, spoons, pieces of
wood, cardboard) were also common.

According to YouTube video descriptions and owner comments,
45
of the individuals performing true tool use were males and
18
were females. No sex information was provided for the remaining
26
individuals. As owners provided no information on whether sex had been
established through genetic testing, and sexual dimorphism in parrots is
rare\textsuperscript{55,56}, we could not typically ascertain if descriptions
were accurate. It is unclear if the disproportionately large number of males in
the sample is a consequence of owners more readily assuming their parrots are
male when they have not been genetically tested, owners being more likely to own
or film male parrots, or male parrots exhibiting more true tool use behaviours
than female parrots.

Figure \ref{fig:plotPhylo1} maps the findings from the video survey onto a
maximum clade credibility phylogeny for the parrot order, plotted alongside
species previously identified in the scientific literature. Before the video
survey, 11 parrot species (3\%) had been identified as tool users in the
scientific literature. Across our video survey, we observed true tool use in
22
species, 5 of which overlapped with the scientific literature and
17
of which were novel species. All of the species identified in the video survey
were cockatoos (\emph{Cacatuidae}), Old World parrots (\emph{Psittacinae}), or neotropical
parrots (\emph{Arinae}). The most common species in our survey, accounting for
41 videos from
37
individuals, was the green-cheeked conure (\emph{Pyrrhura molinae}). In accordance
with the scientific literature, the video survey found no evidence of tool use
in any species of Psittaculidae, despite this family containing some of the most
commonly kept pet species, including lovebirds, lorikeets, and Asian parakeet
species. Combining both the video survey and the scientific literature, we can
thus identify 28 tool-using parrot species overall (7\%), compared to the 11
previously reported.














\begin{figure}
\centering
\includegraphics{manuscript_files/figure-latex/plotPhylo1-1.pdf}
\caption{\label{fig:plotPhylo1}\emph{Results of crowdsourcing video survey and phylogenetic
survival cure modelling mapped onto a maximum clade credibility phylogeny of the
parrot order.} Orange points in the inner ring indicate species observed in the
video survey, with point size scaled by the number of videos for each species
(note that three species observed only in the video survey are not present in
the phylogeny due to a lack of genomic data: \emph{Psittacara erythrogenys},
\emph{Psittacus timneh}, and \emph{Aratinga nenday}). Blue points in the outer ring
indicate species observed in the scientific literature, with point size scaled
by the number of papers published on each species. Yellow species names indicate
the top ten most likely tool-using species from our phylogenetic survival cure
model which were not observed in the scientific literature or the video survey.
Total \emph{n} = 174 species.}
\end{figure}

The identification of new tool-using species in our video survey increases the
extent to which phylogeny can explain the distribution of tool use in the parrot
order. We estimated phylogenetic signal (Pagel's \(\lambda\)) of tool use using
both the pre-video-survey and post-video-survey data. Pagel's \(\lambda\) varies
between 0 and 1, where 0 implies that the distribution of a trait across species
is unexplained by phylogenetic relatedness and 1 implies that the distribution
of a trait across species is fully explained by phylogeny. Using the evidence of
tool use from the scientific literature alone (pre-video-survey data; 11
tool-using species), we estimated an average posterior Pagel's \(\lambda\) of
0.60 (95\% credible interval {[}0.00
0.90{]}; total \emph{n} =
174 species). This estimate was moderate-to-strong, but highly
uncertain. In comparison, combining the evidence from both the literature and
the video survey (post-video-survey data; 28 tool-using species) resulted in a
stronger and more certain estimate of phylogenetic signal. With these data, we
estimated Pagel's \(\lambda\) = 0.65 (95\% CI
{[}0.50
0.77{]}; total \emph{n} =
174 species). Thus, the results of our video survey increase
the extent to which the distribution of tool use across parrot species can be
explained by shared phylogenetic ancestry. This suggests that we can potentially
use phylogenetic information, along with other variables, to identify further
tool-using parrot species that may remain undetected.

\hypertarget{phylogenetic-survival-cure-modelling-identifies-further-candidate-tool-users}{%
\subsection{Phylogenetic survival cure modelling identifies further candidate tool users}\label{phylogenetic-survival-cure-modelling-identifies-further-candidate-tool-users}}

In addition to the 28 tool-using species identified in the literature and our
video survey, we fitted a Bayesian phylogenetic survival cure model to rank
further species that are likely to be undetected tool-users (i.e., tool-using
species with no tool use reported in the literature or in crowdsourced videos).

Survival cure models\textsuperscript{57}, also known as split population
models\textsuperscript{58}, are used to analyse the time to some event of interest
with the added assumption that a certain proportion of the population will never
experience the event, no matter how long they are measured for. These models
have been used to analyse a variety of right-censored outcomes, from cancer
relapse\textsuperscript{57} to criminal recidivism\textsuperscript{58}. The data are
right-censored because some individuals will have experienced the event when
they are measured (e.g., disease onset, return to prison) while others will have
not experienced the event. For those who have not experienced the event, this
may be because (\emph{i}) the event has not happened to them yet or (\emph{ii}) the event
will \emph{never} happen to them. Survival cure models treat these two processes
separately.

Our tool use problem has the same features. We are modelling a time-to-event;
specifically, the amount of ``time'' (i.e., observation opportunities measured as
the number of published papers or crowdsourced videos) until tool use is
identified. This is right-censored data because many species will not have had
tool use identified when we measure them. Moreover, we can assume that a certain
proportion of the population will never experience the event -- that is, they
are a non-tool-user and so we will never identify tool use no matter how long we
measure them for.

In our model, we infer the tool-using status of each species by allowing each
species to have their own probability of being a non-tool-user. Following our
causal model (Figure \ref{fig:plotDAG}), we predict these probabilities based
on feeding strategy, encephalisation quotient, and phylogenetic history (see
Methods for full model). The model additionally takes research effort into
account by allowing that, among species for which tool use is unobserved, all
else being equal those with fewer published papers and fewer video search hits
have a higher probability of being undetected tool users (Supplementary Figure
\ref{fig:plotSurvCure3}).

We found that this phylogenetic survival cure model was able to adequately
distinguish between species with and without evidence for tool use, with an
area-under-the-curve classification statistic of
0.95 (Supplementary Figure
\ref{fig:plotSurvCure9}). To further estimate the accuracy of the model's
predictions, we also used a leave-one-species-out approach with known tool
users. For each of the 25 tool-using species that were represented on the
phylogeny and for which we had brain size and genomic data (we lacked data for
three tool-using species), we fitted the model to a modified dataset which set
tool use to be absent for the target species in both the scientific literature
and the video survey. Across 25 cross-validation models,
18 models (72\%)
continued to predict the target species as having a median posterior probability
of tool use that was within the range of all other tool users. This
classification rate was greater than the baseline classification rate of
26\%
for species without evidence of tool use in the full model
(38
of 149 species without
evidence of tool use had a median posterior probability of tool use that was
within the range of the tool-using species). Together, the area-under-the-curve
statistic and the leave-one-species-out approach suggest that the model is able
to adequately classify known tool users, with some error.

Figure \ref{fig:plotSurvCure1} visualises the ranked posterior probabilities of
tool use from the phylogenetic survival cure model for all parrot species. As
expected, the known tool users are ranked towards the top of this list. However,
several ``tool use absent'' species also rank highly on the list, despite not
being identified as tool users in the scientific literature or in our video
survey. In fact, according to the model, the most likely tool user is a species
for which tool use is unobserved in our data: the blue-eyed cockatoo (\emph{Cacatua
opthalmica}). This species is endemic to Papua New Guinea and is relatively
understudied, with only 6 published papers
and 596 video search hits, which is fewer than the
model expects are necessary to discover tool use when it is present (Figure
\ref{fig:plotSurvCure2}). This species is also found in the \emph{Cacatua} genus, a
clade containing several known tool users. This prediction makes sense given the
high phylogenetic signal for tool use reported by the model (Supplementary
Figures \ref{fig:plotSurvCure5} and \ref{fig:plotSurvCure6}). Beyond the
blue-eyed cockatoo, other highly ranked species without observed evidence of
tool use are the Meyer's parrot (\emph{Poicephalus meyeri}), the golden parakeet
(\emph{Guaruba guarouba}), the long-billed corella (\emph{Cacatua tenuirostris}), the
Solomons cockatoo (\emph{Cacatua ducorpsii}), the red-fronted parrot (\emph{Poicephalus
gulielmi}), the Cape parrot (\emph{Poicephalus robustus}), the yellow-eared parrot
(\emph{Ognorhynchus icterotis}), the red-vented cockatoo (\emph{Cacatua haematuropygia}),
and the gang-gang cockatoo (\emph{Callocephalon fimbriatum}). Figure
\ref{fig:plotPhylo1} plots these species on the parrot phylogeny, using the top
ten highest ranked species without observed evidence of tool use as an arbitrary
cutoff for visualisation purposes.






\begin{figure}
\centering
\includegraphics{manuscript_files/figure-latex/plotSurvCure1-1.pdf}
\caption{\label{fig:plotSurvCure1}\emph{Posterior predicted probabilities of tool use for
each species from our phylogenetic survival cure model.} Points are posterior
medians and lines are 50\% and 95\% credible intervals. Total \emph{n} =
174 species.}
\end{figure}






\begin{figure}
\centering
\includegraphics{manuscript_files/figure-latex/plotSurvCure2-1.pdf}
\caption{\label{fig:plotSurvCure2}\emph{Expected number of published papers and videos until
tool use discovery, according to the survival component of the phylogenetic
survival cure model.} Densities are full posterior distributions, points are
posterior medians, and lines are 50\% and 95\% credible intervals.}
\end{figure}

The posterior probabilities shown in Figure \ref{fig:plotSurvCure1} are
estimated with uncertainty, so it is difficult to ``identify'' any particular
species as an undetected tool user. Nevertheless, taking the sum of all the
posterior probabilities for the
149 species without
observed evidence of tool use, we can estimate that around
26 of those species are likely to be
undetected tool users (median sum of probabilities =
25.68, 95\% CI
{[}15.15
41.33{]}). When combined with the
species known to use tools, this implies that between 11\% and 17\% of extant
parrot species may be tool users.

\hypertarget{implications-for-the-evolutionary-drivers-and-origins-of-tool-use}{%
\subsection{Implications for the evolutionary drivers and origins of tool use}\label{implications-for-the-evolutionary-drivers-and-origins-of-tool-use}}

The predicted probabilities from our phylogenetic survival cure model have
implications for inferences about the evolutionary drivers and origins of tool
use in the parrot order. Regarding the drivers of tool use hypothesised in
Figure \ref{fig:plotDAG}, the phylogenetic survival cure model revealed that
encephalisation quotient strongly positively predicted the probability of tool
use (median posterior log odds slope = 1.12, 95\%
CI {[}0.39
2.00{]}; total \emph{n} = 174
species; Figure \ref{fig:plotSurvCure4}). This helps explain the ranking in
Figure \ref{fig:plotSurvCure1}: the blue-eyed cockatoo has the largest relative
brain size in the dataset. We also found that feeding generalist species were
slightly more likely to be tool users, though the posterior difference between
generalists and specialists was quite uncertain (median posterior log odds
difference = 0.33, 95\% CI {[}
-1.13 1.76{]}; total
\emph{n} = 174 species). These results from the survival cure model
differed from the results of models fitted to pre-video-survey and
post-video-survey data without the survival cure component, which found no
effect of relative brain size and no difference between feeding strategies,
respectively (Supplementary Figure \ref{fig:plotComparison}).













\begin{figure}
\centering
\includegraphics{manuscript_files/figure-latex/plotSurvCure4-1.pdf}
\caption{\label{fig:plotSurvCure4}\emph{Posterior predictions for the effects of feeding
strategy and encephalisation quotient on the probability of tool use from the
phylogenetic survival cure model.} In the left plot, points and lines represent
posterior medians and 50\% and 95\% credible intervals, with densities
representing full posterior distributions. In the right plot, the line and
shaded areas represent the posterior median regression line with 50\% and 95\%
credible intervals. In both plots, individual species are coloured according to
the presence / absence of tool use in the video survey and the scientific
literature. Total \emph{n} = 174 species, generalist \emph{n} =
121 species, specialist \emph{n} =
53 species.}
\end{figure}

Regarding the origins of tool use, we fitted exploratory ancestral state
reconstruction models to the pre-video-survey data, the post-video-survey data,
and the predicted probabilities from the phylogenetic survival cure model. The
discoveries from our video survey and from our phylogenetic modelling increased
the likelihood that tool use was present in the most recent common ancestor for
several parrot genera. These include the amazon parrots native to the Americas
(\emph{Amazona}), the true white cockatoos and corellas found in South East Asia and
Australasia (\emph{Cacatua}), the kea and the kākā from New Zealand (\emph{Nestor}), and
the \emph{Poicephalus} genus native to Africa (Table \ref{tab:tableASR};
Supplementary Figures \ref{fig:plotASR1} - \ref{fig:plotASR3}). These findings
suggest that species from each of these genera may share their tool use
capabilities via common descent from their respective common ancestors, rather
than via independent evolution within each genus or behavioural innovation
within species.







\begin{table}[tbp]

\begin{center}
\begin{threeparttable}

\caption{\label{tab:tableASR}Estimated probabilities of tool use for most recent common
ancestors of several parrot genera. \emph{Probabilities estimated using exploratory
ancestral state reconstruction models fitted to the pre-video-survey data,
post-video-survey data, and predicted probabilities from the phylogenetic
survival cure model. Total} n \emph{= 174 species.}}

\begin{tabular}{llll}
\toprule
Genus & \multicolumn{1}{c}{Pre-video-survey} & \multicolumn{1}{c}{Post-video-survey} & \multicolumn{1}{c}{Survival cure probabilities}\\
\midrule
\textit{Amazona} & 0.06, 95\% CI [0.01 0.17] & 0.21, 95\% CI [0.04 0.66] & 0.48, 95\% CI [0.05 0.84]\\
\textit{Cacatua} & 0.08, 95\% CI [0.03 0.20] & 0.17, 95\% CI [0.03 0.45] & 0.81, 95\% CI [0.23 0.99]\\
\textit{Nestor} & 0.11, 95\% CI [0.05 0.29] & 0.21, 95\% CI [0.12 0.37] & 0.53, 95\% CI [0.29 0.66]\\
\textit{Poicephalus} & 0.06, 95\% CI [0.02 0.11] & 0.19, 95\% CI [0.12 0.41] & 0.72, 95\% CI [0.16 0.90]\\
\bottomrule
\end{tabular}

\end{threeparttable}
\end{center}

\end{table}

\hypertarget{discussion}{%
\section{Discussion}\label{discussion}}

Since the earliest anecdotes of parrots using tools by Wallace in the 1880s\textsuperscript{59} and more systematic anecdotal reports in the 1970s\textsuperscript{60},
only 11 parrot species (3\% of all extant parrots) have been documented as tool
users in the scientific literature. Our study used crowdsourcing to identify 17
additional tool-using parrot species that are new to science, more than doubling
the overall count to 28 species (7\%). These species consisted of cockatoos
(\emph{Cacatuidae}), Old World parrots (\emph{Psittacinae}), and neotropical parrots
(\emph{Arinae}). Beyond the crowdsourcing survey, the strong phylogenetic signal in
our dataset allowed us to use phylogenetic information, along with other
variables, to infer the unobserved probabilities of tool use across the parrot
order. Our phylogenetic survival cure model incorporated information on
phylogenetic history, research effort, relative brain size, and feeding
specialisation to rank parrot species that were most likely to be undetected
tool users. The sum of probabilities from this model implied that between 15
and 41 of the species without observed evidence of tool use are likely to be
undetected tool users, suggesting that the true proportion of tool users may be
as high as 17\%.

These findings have a number of important implications. First, our findings show
that current research effort in the scientific literature is insufficient to
capture the real world occurrence of parrot tool use. If the scientific
literature had sampled the natural world sufficiently, we would expect to see
close correspondence between those species reported as tool users in the
literature and those species the public have filmed performing tool use.
Instead, we discovered a large discrepancy between these two data sources, both
in the prevalence of tool use and the species identified. This raises the
possibility that insufficient research effort is a general issue across the
scientific literature, both for tool use in other groups and for other rare
behaviours.

Second, in terms of the evolution of tool use in parrots, our study challenges a
key assumption made in the literature to date: that only a minority of parrots
are tool users\textsuperscript{30,31,34,37,38}.
The paucity of evidence for tool use across parrots in the literature initially
implied that the capacity for tool use may have evolved independently in
different parrot species. Our discovery of the widespread distribution of tool
use across the parrot phylogeny, along with the strong phylogenetic signal in
this expanded dataset, challenges this and suggests that, at least for some
parrot clades, the capacity for tool use might be a homologous trait that has
been evolutionarily conserved. Our exploratory ancestral state reconstruction
analysis provides preliminary support for this hypothesis, revealing an
increased probability of tool use among the most recent common ancestors for the
\emph{Amazona}, \emph{Cacatua}, \emph{Nestor}, and \emph{Poicephalus} genera. Even at this
preliminary stage, our analysis therefore raises an alternative hypothesis for
the observed tool use in \emph{Cacatua}\textsuperscript{30,38} and
\emph{Nestor}\textsuperscript{26--29,31,34}, namely that tool behaviours have arisen due to the
common ancestor having the capacity to use tools, rather than from independent
evolution or behavioural innovation within species.

Third, our results support existing theories of the drivers of tool use. We
found that encephalisation was strongly positively related to the probability of
tool use in our phylogenetic model, supporting previous theories linking
relative brain size to increased tool innovation in birds\textsuperscript{11--13} and primates\textsuperscript{61}. We acknowledge that
encephalisation quotient is not a perfect measure due to measurement error and
challenges with interpretation\textsuperscript{3,4}. However, encephalisation
quotient has much greater coverage across the parrot phylogeny than more
fine-grained measures like whole neuron count\textsuperscript{62} and there is no reason
to think that measurement error would produce the consistent patterns across our
study and prior work. To understand the causal mechanisms responsible for these
relationships, we encourage further work on the specific neural correlates of
technical intelligence in parrots, e.g.,\textsuperscript{63}. In our phylogenetic
model, we also found that tool use was somewhat more likely among feeding
generalists compared to feeding specialists, although this difference was
uncertain. This trend supports previous suggestions that increased cognitive
abilities and technical innovation rates are required to expand a generalist
species' dietary niche\textsuperscript{16,18,64}. However, the trend contradicts theories linking tool use to
dietary specialisation, whereby species eating specific foods that require
extractive foraging have higher cognitive ability and are especially prone to
using tools\textsuperscript{18}.

All instances of tool use in our crowdsourcing survey met the established
criteria for tool use in the literature: the manipulation of an unattached
object as an extension of the animal's body to achieve a goal\textsuperscript{10}.
However, a striking feature of the dataset is that the 17 additional species
identified in the video survey were exclusively observed using tools for
self-scratching. While parrots in the wild use tools to achieve a variety of
goals, including extractive foraging and preening\textsuperscript{34,38},
parrots in captivity only require tools to achieve the latter goal. Despite this
difference in the type of tool use observed, self-scratching is nonetheless a
major category of animal tool use. Self-scratching also meets the definition for
a more complex type of embodied tool use known as `tooling', the deliberate
generation of a mechanical interface by using an object to manipulate another
target or surface\textsuperscript{65}. The fact that we found strong phylogenetic
clustering of self-scratching tool use from the video survey with other examples
of tool use from the literature supports a common underlying cognitive
mechanism. In line with this, some of the species of parrots that use tools for
self-scratching in captivity also use tools for other purposes in the wild\textsuperscript{30,31,34,38}.

There was little evidence to suggest that the observations of self-scratching
tool use in the video survey were merely unintentional accidents or explicitly
trained behaviours. Individual parrots often used tools slowly and repetitively
over long periods of time, even across multiple different videos, suggesting
that their behaviour was not random or accidental\textsuperscript{66}. Parrots
preferentially employed self-scratching tools on areas of their body that were
otherwise inaccessible, with 96\% of all instances involving scratching of the
head or neck, suggesting intentional tool use. For 60\% of the species in the
video survey, we found two or more videos of repetitive and sustained scratching
by different individuals of the same species in separate households, suggesting
that the manipulations were intentional and recurring events that did not
represent unusual stereotypies of any single individual. Regarding the
possibility of training or cueing from owners, over half of the videos contained
no evidence of human interaction aside from filming the behaviour. Humans only
handed parrots their tools in two of the videos, and none of the videos featured
owners directly rewarding tool use behaviours with food. Finally, the high
levels of phylogenetic signal in our data provide strong evidence that the
observations from our video survey reflect biologically-endowed capacities for
tool use rather than accidental or trained behaviours, which would likely appear
uniformly across the phylogeny.

While missing data imputation is becoming more common in phylogenetic
analyses\textsuperscript{67}, the important distinction between absence of
evidence and evidence of absence has not been given as much attention. Our
phylogenetic analysis provides one approach to this problem by distinguishing
between true absences of tool use and absences of tool use due to a lack of
research effort in the scientific literature or in crowdsourced videos. To
achieve this, we explicitly modelled the measurement of the outcome variable
along a research effort time series, such that species with lower research
effort in the literature or in videos were likely to be censored. In line with
our causal model, we also included relative brain size, feeding strategy, and
phylogenetic history as predictors of unobserved tool use. We encourage
researchers to test this model by directing future study efforts towards the
parrot species with the highest probabilities of being undetected tool-users.
Future work should also refine the causal model in Figure \ref{fig:plotDAG} to
provide more certain estimates of tool use probabilities, either by including
additional predictor variables or modelling further causes of measurement error
in the taxonomic record, such as species abundance and geographic
accessibility\textsuperscript{20}.

In conclusion, we have shown that the scientific literature has insufficiently
captured the full distribution of tool use in the parrot order. Our
crowdsourcing survey more than doubled the number of known tool using parrot
species from 11 to 28, and our phylogenetic model suggested that the true
proportion of parrot tool users could be as high as 17\% of all species in this
order. These discoveries have implications for theories of the evolutionary
drivers and origins of tool use in parrots. Beyond parrot tool use, the
crowdsourcing and phylogenetic methods used in this study have the potential to
be applied to other rarely observed behaviours, including tool use in other
taxa\textsuperscript{68}, rhythmic entrainment in birds\textsuperscript{69--72}, teasing behaviours in primates\textsuperscript{73}, and
tactical deception across all animals\textsuperscript{74--76}.
We hope that these methods will continue to uncover a diverse array of ephemeral
behaviours that have as yet gone undetected in the scientific literature.

\hypertarget{methods}{%
\section{Methods}\label{methods}}

\hypertarget{video-searches-and-coding}{%
\subsection{Video searches and coding}\label{video-searches-and-coding}}

Our video search was conducted on YouTube in July 2020. Search terms included
``parrot using tool'' and variants (e.g., ``macaw using tool'', ``lorikeet using
tool'', ``parakeet using tool''), ``tool use in parrot'', ``parrot tool use'', ``parrot
scratching itself'' (included after we found several videos demonstrating
self-care tool use in previous searches) and equivalent terms (e.g., ``parrot
preening itself'', ``parrot grooming itself'', ``parrot scratching''). For all
species that did not display results including object manipulation or scratching
behaviours, we also searched the species' common name(s) + ``tool use'', as well
as the species' common name(s) + ``scratching''. We also searched for translations
of the terms ``parrot tool use'' and ``parrot scratching'' in languages for all
countries where bird ownership was reported as \textgreater5\%\textsuperscript{77}, namely,
Turkish, Czech, Polish, French, Italian, Dutch, German, Russian, Spanish,
Portuguese, and Mandarin.

When we found a relevant video, we also searched for similar content uploaded by
the same person/channel. For each YouTube search conducted, we watched all
relevant videos until we reached five consecutive videos that did not feature
any parrots. At this point, we ended that search and initiated the next search.
In line with previous recommendations\textsuperscript{21}, we planned to exclude any
videos that consisted of four or more shots edited together so as to ensure the
behaviours being observed were not edited or manipulated, but none of the videos
obtained qualified for exclusion.

All videos featuring parrots manipulating objects were investigated for
potential tool use or borderline tool use. We defined tool use as the
manipulation of an unattached object as an extension of the beak or foot to
achieve a goal towards another object, individual, or oneself\textsuperscript{10}.
Borderline tool use was similarly defined, except that it involved the use of an
object that was still attached to a substrate\textsuperscript{54}. For example, if
individuals used a fallen feather or stick for self-scratching this was defined
as tool use, but using one's currently attached tail feathers or cage
furnishings for the same purpose was defined as borderline tool use.
Self-scratching had to involve slow and repeated movements of touching an object
to one's body (or, in the case of borderline tool use, rubbing repetitively
against an attached object\textsuperscript{66}).

All relevant videos were coded for video length, species, tool use presence
(yes/borderline), tool use type (e.g., scratching, feeding), the object being
used (e.g., feather, stick), tool use target, human interaction
(talking or handing object to parrot, holding parrot), and the number of shots
within each video. Our complete dataset also includes the name for each video,
link, subject name, sex (as declared by owner, as most parrot species are not
sexually dimorphic), publishing date, and dates found and coded.

\hypertarget{data-for-parrot-species}{%
\subsection{Data for parrot species}\label{data-for-parrot-species}}

We collected data for 194 parrot species (Supplementary Figures
\ref{fig:plotPhylo2} - \ref{fig:plotPhylo4}). We gathered feeding strategy
data as a dichotomous variable (``generalist'' or ``specialist'') from the
EltonTraits ecological database\textsuperscript{52}. As per the database, specialists
were defined as species whose diet comprised at least 70\% of a single food
source. To calculate relative brain size, we collated data from the literature
for all known body mass (g) and brain mass (g) values across
parrots\textsuperscript{46--51}. For all species for which we obtained body and brain mass data, we
calculated the encephalisation quotient (EQ) using the following
formula\textsuperscript{78}: \(BrainWeight / (0.12 * BodyWeight(\frac{2}{3}))\). We
found body mass and brain mass data for a total of 194 parrot species. This
included all tool-using species in our video dataset, with the exception of
three species: \emph{Diopsittaca nobilis}, \emph{Psittacara erythrogenys}, and \emph{Coracopsis
vasa}. For the latter, we used values for the closely related \emph{Coracopsis
nigra}. The other two species were excluded from the final dataset.

For modelling purposes, we coded research effort in both the scientific
literature and the crowdsourced videos. For the scientific literature, we
operationalised research effort as the number of papers published for each
species' Latin name up to and including the first paper containing tool use for
that species. If no tool use had been identified in the scientific literature
for a species, then we coded the total number of papers published to date. We
used the scientific database Scopus for coding the number of published papers.
For the crowdsourced videos, we coded research effort as the number of search
hits for each species on YouTube. If tool use had been identified on YouTube, we
estimated the number of search hits when the first video of tool use was
published on YouTube, assuming linear growth of search hits since the inception
of YouTube. If tool use had not been identified, we used the current number of
search hits.

For phylogenetic data, we used the phylogenetic tool at www.birdtree.org\textsuperscript{53} to compile 1000 posterior draws of phylogenetic trees for 174 of the
194 parrot species for which both EQ and genomic data exist. A single maximum
clade credibility tree was generated from these posterior draws for
visualisation purposes. In our analyses, we iterated over posterior draws of the
phylogeny to account for phylogenetic uncertainty.

\hypertarget{phylogenetic-signal}{%
\subsection{Phylogenetic signal}\label{phylogenetic-signal}}

We used the \emph{fitDiscrete} function in the \emph{ape} R package\textsuperscript{79} to
calculate phylogenetic signal, for both the pre-survey and post-survey tool use
data. We iterated the model over 100 posterior parrot phylogenies to incorporate
phylogenetic uncertainty.

\hypertarget{causal-model-of-tool-use}{%
\subsection{Causal model of tool use}\label{causal-model-of-tool-use}}

To infer unobserved probabilities of tool use across parrots, we proposed a
causal model of observed tool use (Figure \ref{fig:plotDAG}). We assumed that
observed tool use in the scientific literature and in the crowdsourced videos is
caused by both the unobserved presence or absence of tool use and research
effort, proxied by the number of papers published on a species and the number of
videos published on a species. Tool users are more likely to be observed if they
are well studied, but understudied tool users may go undetected. In addition,
based on theory, we also assumed that unobserved tool use is caused by feeding
strategy and relative brain size\textsuperscript{11--14,16--18}. Finally, we
assumed that shared phylogenetic history causes unobserved confounding and
non-independence in unobserved tool use, feeding strategy, and relative brain
size across the parrot phylogeny.

\hypertarget{bayesian-phylogenetic-survival-cure-model}{%
\subsection{Bayesian phylogenetic survival cure model}\label{bayesian-phylogenetic-survival-cure-model}}

Given our proposed causal model, we constructed a statistical model to impute
unobserved probabilities of tool use and test existing theories of the evolution
of tool use in parrots. To understand the model, suppose that we have the
following observed variables for parrot species \(i\). For the scientific
literature, we declare \(\text{N}_{\text{Lit},i}\) as the number of papers
published before and up to tool use identification for species \(i\) (or, if tool
use has not been identified, the total number of papers published for species
\(i\)) and \(\text{T}_{\text{Lit},i}\) as a binary variable stating whether (1) or
not (0) tool use has yet been observed in the scientific literature for species
\(i\). For the crowdsourced videos, we declare \(\text{N}_{\text{Vid},i}\) as the
number of videos published before and up to tool use identification for species
\(i\) (or, if tool use has not been identified, the total number of videos
published for species \(i\)) and \(\text{T}_{\text{Vid},i}\) as a binary variable
stating whether (1) or not (0) tool use has yet been observed in the
crowdsourced videos for species \(i\). Additionally, \(\text{F}_i\) and
\(\text{EQ}_i\) are feeding strategy and encephalisation quotient values for
species \(i\) and we have a phylogenetic distance matrix \(D\) that describes the
patristic distances between all parrot species.

We assume that species \(i\) is a non-tool-user with some
probability \(p_i\). We also assume that tool use is identified in the scientific
literature and the crowdsourced videos at constant rates \(\lambda_{\text{Lit}}\)
and \(\lambda_{\text{Vid}}\) following exponential survival functions. Given these
assumptions, we can then describe the different ways in which variables
\(\text{N}_\text{Lit}\) and \(\text{N}_\text{Vid}\) can be distributed. Focusing on
the scientific literature, if tool use has been observed
(\(\text{T}_{\text{Lit},i} = 1\)), then the likelihood for
\(\text{N}_{\text{Lit},i}\) is:

\begin{align}
\text{Pr}(\text{N}_{\text{Lit},i}|\text{T}_{\text{Lit},i} = 1,p_i,\lambda_\text{Lit}) = (1-p_i)\cdot\text{Exponential}(\text{N}_{\text{Lit},i}|\lambda_\text{Lit})
\end{align}

On the other hand, if tool use has not yet been observed
(\(\text{T}_{\text{Lit},i} = 0\)), there are two ways that the outcome variable
could have been realised. First, the species could be a non-tool-user with
probability \(p_i\). Second, the species could be a tool-user with probability
(\(1 - p_i\)) that has been censored and has not had its tool use measured yet.
Together, then, the likelihood for \(\text{N}_{\text{Lit},i}\) is:

\begin{align}
\text{Pr}(\text{N}_{\text{Lit},i}|\text{T}_{\text{Lit},i} = 0,p_i,\lambda_\text{Lit}) = p_i + ((1-p_i)\cdot\text{Exponential-CCDF}(\text{N}_{\text{Lit},i}|\lambda_\text{Lit}))
\end{align}

The Exponential-CCDF function allows for the censored nature of the data. The
same data generating process is assumed to underlie the crowdsourced videos.

We define the mixture likelihood \(\text{SurvivalCure}\) as the distribution
above, with parameters \(p\) (the probability of being a non-tool-user)
and \(\lambda\) (the rate of the exponential distribution). We use an
Ornstein-Uhlenbeck Gaussian process\textsuperscript{80} to model phylogenetic
covariance. Below, we specify the full model with priors:

\begin{align}
N_{\text{Lit},i} &\sim \text{SurvivalCure}(\lambda_{\text{Lit},i}, p_{i}) \\
N_{\text{Vid},i} &\sim \text{SurvivalCure}(\lambda_{\text{Vid},i}, p_{i}) \nonumber \\
\lambda_{\text{Lit},i} &= 1 / \text{exp}(\gamma_\text{Lit}) \nonumber \\
\lambda_{\text{Vid},i} &= 1 / \text{exp}(\gamma_\text{Vid}) \nonumber \\
\text{logit}(p_{i}) &= \alpha_{\text{FEEDING}[i]} + \beta\text{EQ}_{i} +  k_{\text{SPECIES}[i]} \nonumber \\
\begin{pmatrix}k_{1}\\k_{2}\\...\\k_{n}\\\end{pmatrix}
&\sim \text{MVNormal}
\begin{pmatrix}\begin{pmatrix}0\\0\\...\\0\\\end{pmatrix},\bf{K}\end{pmatrix} \nonumber \\
\bf{K}_{ij} &= \eta^2\text{exp}(-\rho^2D_{ij}) \nonumber \\
\gamma_\text{Lit}, \gamma_\text{Vid}, \alpha_{1,2}, \beta &\sim \text{Normal(0, 1)} \nonumber \\
\eta^2, \rho^2 &\sim \text{Exponential(0.5)} \nonumber
\end{align}

The priors for this model produce reasonable prior predictions of the
probabilities of tool use for each parrot species (Supplementary Figure
\ref{fig:plotSurvCure7}), but a sensitivity analysis revealed that the ranking
and posterior probabilities reported in the main text were robust to modifying
these priors (Supplementary Figure \ref{fig:plotSurvCure8}). We estimated the
posterior distribution of this model using Hamiltonian Monte Carlo as
implemented in Stan version 2.26.1\textsuperscript{81}. We ran the model for 4000
samples, with 2000 warmup samples, and iterated the model over 100 posterior
parrot phylogenies to incorporate phylogenetic uncertainty. R-hat values and
effective sample sizes suggested that the model converged normally. Trace plots
are reported in Supplementary Figure \ref{fig:plotTrace}. We report
equal-tailed credible intervals to describe the posterior distribution of this
model in the main text.

To validate our method, we fitted the model to 100 simulated datasets with known
parameters. The model was able to successfully recover those parameters
(Supplementary Figure \ref{fig:plotSurvCureSim}). We also ran a
leave-one-species-out exercise to ensure that we could accurately predict known
tool users. We repeated this approach for each known tool user by setting
observed tool use to zero. Cross-validation results are reported in the main
text.

\hypertarget{ancestral-state-reconstruction}{%
\subsection{Ancestral state reconstruction}\label{ancestral-state-reconstruction}}

To determine whether the identification of novel tool-using species has
implications for our understanding of the evolutionary origins of tool use in
parrots, we fitted three exploratory ancestral state reconstruction models. We
used the \emph{ancThresh} function from the \emph{phytools} R package\textsuperscript{82},
iterating the function over 100 posterior parrot phylogenies. This function
estimates discrete ancestral states by assuming the evolution of a latent
continuous variable following an Ornstein-Uhlenbeck process. We fitted this
model to three different outcome variables: (\emph{i}) presence vs.~absence of tool
use in scientific literature only, (\emph{ii}) presence vs.~absence of tool use in
literature and/or videos, and (\emph{iii}) the median predicted probabilities of tool
use from the phylogenetic survival cure model.

\hypertarget{reproducibility}{%
\subsection{Reproducibility}\label{reproducibility}}

All analyses were conducted in R v4.2.1.\textsuperscript{83}. Visualisations were
produced using the \emph{ggtree}\textsuperscript{84}, \emph{ggplot2}\textsuperscript{85}, and \emph{cowplot}\textsuperscript{86} packages. The manuscript was reproducibly generated using the
\emph{targets}\textsuperscript{87} and \emph{papaja}\textsuperscript{88} packages. Code to reproduce all
analyses and figures can be found here:
\url{https://github.com/ScottClaessens/phyloParrot}

\newpage
\nolinenumbers

\hypertarget{acknowledgements}{%
\section{Acknowledgements}\label{acknowledgements}}

This project was made possible through the support of a grant from the Templeton
World Charity Foundation (A.H.T., X.J.N.). The authors would like to thank
Daniel Sol for providing feedback on a previous version of the manuscript.

\hypertarget{author-contributions}{%
\section{Author Contributions}\label{author-contributions}}

All authors contributed to the conceptualisation of the paper. A.P.M.B., X.J.N.,
and A.H.T. developed the video search methodology. S.C., D.W., and Q.A.D.
developed the statistical models and analysed the data. All the authors wrote
the manuscript and approved the final version for submission.

\hypertarget{competing-interests}{%
\section{Competing Interests}\label{competing-interests}}

The authors declare no competing interests.

\hypertarget{data-availability}{%
\section{Data Availability}\label{data-availability}}

All data used in this study are publicly available on GitHub:
\url{https://github.com/ScottClaessens/phyloParrot}

\hypertarget{code-availability}{%
\section{Code Availability}\label{code-availability}}

All code to reproduce the analyses in this study are publicly available on
GitHub: \url{https://github.com/ScottClaessens/phyloParrot}

\newpage

\hypertarget{references}{%
\section{References}\label{references}}

\begingroup

\hypertarget{refs}{}
\begin{CSLReferences}{0}{0}
\leavevmode\vadjust pre{\hypertarget{ref-MacLean2012}{}}%
\CSLLeftMargin{1. }%
\CSLRightInline{MacLean, E. L. \emph{et al.} \href{https://doi.org/10.1007/s10071-011-0448-8}{How does cognition evolve? Phylogenetic comparative psychology}. \emph{Animal Cognition} \textbf{15}, 223--238 (2012).}

\leavevmode\vadjust pre{\hypertarget{ref-Lefebvre2011}{}}%
\CSLLeftMargin{2. }%
\CSLRightInline{Lefebvre, L. \href{https://doi.org/10.1098/rsbl.2010.0556}{Taxonomic counts of cognition in the wild}. \emph{Biology Letters} \textbf{7}, 631--633 (2011).}

\leavevmode\vadjust pre{\hypertarget{ref-Logan2018}{}}%
\CSLLeftMargin{3. }%
\CSLRightInline{Logan, C. J. \emph{et al.} \href{https://doi.org/10.3819/CCBR.2018.130008}{Beyond brain size: Uncovering the neural correlates of behavioral and cognitive specialization}. \emph{Comparative Cognition \& Behavior Reviews} \textbf{13}, 55--89 (2018).}

\leavevmode\vadjust pre{\hypertarget{ref-Healy2007}{}}%
\CSLLeftMargin{4. }%
\CSLRightInline{Healy, S. D. \& Rowe, C. \href{https://doi.org/10.1098/rspb.2006.3748}{A critique of comparative studies of brain size}. \emph{Proceedings of the Royal Society B: Biological Sciences} \textbf{274}, 453--464 (2007).}

\leavevmode\vadjust pre{\hypertarget{ref-Powell2017}{}}%
\CSLLeftMargin{5. }%
\CSLRightInline{Powell, L. E., Isler, K. \& Barton, R. A. \href{https://doi.org/10.1098/rspb.2017.1765}{Re-evaluating the link between brain size and behavioural ecology in primates}. \emph{Proceedings of the Royal Society B: Biological Sciences} \textbf{284}, 20171765 (2017).}

\leavevmode\vadjust pre{\hypertarget{ref-Goodall1964}{}}%
\CSLLeftMargin{6. }%
\CSLRightInline{Goodall, J. Tool-using and aimed throwing in a community of free-living chimpanzees. \emph{Nature} \textbf{201}, 1264--1266 (1964).}

\leavevmode\vadjust pre{\hypertarget{ref-Hunt1996}{}}%
\CSLLeftMargin{7. }%
\CSLRightInline{Hunt, G. R. \href{https://doi.org/10.1038/379249a0}{Manufacture and use of hook-tools by {N}ew {C}aledonian crows}. \emph{Nature} \textbf{379}, 249--251 (1996).}

\leavevmode\vadjust pre{\hypertarget{ref-Smolker1997}{}}%
\CSLLeftMargin{8. }%
\CSLRightInline{Smolker, R., Richards, A., Connor, R., Mann, J. \& Berggren, P. \href{https://doi.org/10.1111/j.1439-0310.1997.tb00160.x}{Sponge carrying by dolphins ({D}elphinidae, \emph{{T}ursiops} sp.): A foraging specialization involving tool use?} \emph{Ethology} \textbf{103}, 454--465 (1997).}

\leavevmode\vadjust pre{\hypertarget{ref-Finn2009}{}}%
\CSLLeftMargin{9. }%
\CSLRightInline{Finn, J. K., Tregenza, T. \& Norman, M. D. \href{https://doi.org/10.1016/j.cub.2009.10.052}{Defensive tool use in a coconut-carrying octopus}. \emph{Current Biology} \textbf{19}, R1069--R1070 (2009).}

\leavevmode\vadjust pre{\hypertarget{ref-Shumaker2011}{}}%
\CSLLeftMargin{10. }%
\CSLRightInline{Shumaker, R. W., Walkup, K. R., Beck, B. B. \& Burghardt, G. M. \emph{Animal tool behavior: The use and manufacture of tools by animals}. (Johns Hopkins University Press). doi:\href{https://doi.org/10.1353/book.98237}{10.1353/book.98237}.}

\leavevmode\vadjust pre{\hypertarget{ref-Lefebvre2002}{}}%
\CSLLeftMargin{11. }%
\CSLRightInline{Lefebvre, L., Nicolakakis, N. \& Boire, D. \href{https://doi.org/10.1163/156853902320387918}{Tools and brains in birds}. \emph{Behaviour} \textbf{139}, 939--973 (2002).}

\leavevmode\vadjust pre{\hypertarget{ref-Lefebvre1997}{}}%
\CSLLeftMargin{12. }%
\CSLRightInline{Lefebvre, L., Whittle, P., Lascaris, E. \& Finkelstein, A. \href{https://doi.org/10.1006/anbe.1996.0330}{Feeding innovations and forebrain size in birds}. \emph{Animal Behaviour} \textbf{53}, 549--560 (1997).}

\leavevmode\vadjust pre{\hypertarget{ref-Lefebvre2004}{}}%
\CSLLeftMargin{13. }%
\CSLRightInline{Lefebvre, L., Reader, S. M. \& Sol, D. \href{https://doi.org/10.1159/000076784}{Brains, innovations and evolution in birds and primates}. \emph{Brain, Behavior and Evolution} \textbf{63}, 233--246 (2004).}

\leavevmode\vadjust pre{\hypertarget{ref-Sol2005}{}}%
\CSLLeftMargin{14. }%
\CSLRightInline{Sol, D., Duncan, R. P., Blackburn, T. M., Cassey, P. \& Lefebvre, L. \href{https://doi.org/10.1073/pnas.0408145102}{Big brains, enhanced cognition, and response of birds to novel environments}. \emph{Proceedings of the National Academy of Sciences} \textbf{102}, 5460--5465 (2005).}

\leavevmode\vadjust pre{\hypertarget{ref-Sol2009}{}}%
\CSLLeftMargin{15. }%
\CSLRightInline{Sol, D. \href{https://doi.org/10.1098/rsbl.2008.0621}{Revisiting the cognitive buffer hypothesis for the evolution of large brains}. \emph{Biology Letters} \textbf{5}, 130--133 (2009).}

\leavevmode\vadjust pre{\hypertarget{ref-Ducatez2015}{}}%
\CSLLeftMargin{16. }%
\CSLRightInline{Ducatez, S., Clavel, J. \& Lefebvre, L. \href{https://doi.org/10.1111/1365-2656.12255}{Ecological generalism and behavioural innovation in birds: Technical intelligence or the simple incorporation of new foods?} \emph{Journal of Animal Ecology} \textbf{84}, 79--89 (2015).}

\leavevmode\vadjust pre{\hypertarget{ref-Overington2011}{}}%
\CSLLeftMargin{17. }%
\CSLRightInline{Overington, S. E., Griffin, A. S., Sol, D. \& Lefebvre, L. \href{https://doi.org/10.1093/beheco/arr130}{Are innovative species ecological generalists? A test in {N}orth {A}merican birds}. \emph{Behavioral Ecology} \textbf{22}, 1286--1293 (2011).}

\leavevmode\vadjust pre{\hypertarget{ref-HenkeVonDerMalsburg2020}{}}%
\CSLLeftMargin{18. }%
\CSLRightInline{Henke-von der Malsburg, J., Kappeler, P. M. \& Fichtel, C. Linking ecology and cognition: Does ecological specialisation predict cognitive test performance? \emph{Behavioral Ecology and Sociobiology} \textbf{74}, 154 (2020).}

\leavevmode\vadjust pre{\hypertarget{ref-MettkeHofmann2014}{}}%
\CSLLeftMargin{19. }%
\CSLRightInline{Mettke-Hofmann, C. \href{https://doi.org/10.1002/wcs.1289}{Cognitive ecology: Ecological factors, life-styles, and cognition}. \emph{WIREs Cognitive Science} \textbf{5}, 345--360 (2014).}

\leavevmode\vadjust pre{\hypertarget{ref-Ducatez2014}{}}%
\CSLLeftMargin{20. }%
\CSLRightInline{Ducatez, S. \& Lefebvre, L. \href{https://doi.org/10.1371/journal.pone.0089955}{Patterns of research effort in birds}. \emph{PLOS ONE} \textbf{9}, e89955 (2014).}

\leavevmode\vadjust pre{\hypertarget{ref-Nelson2013}{}}%
\CSLLeftMargin{21. }%
\CSLRightInline{Nelson, X. J. \& Fijn, N. \href{https://doi.org/10.1016/j.anbehav.2012.12.009}{The use of visual media as a tool for investigating animal behaviour}. \emph{Animal Behaviour} \textbf{85}, 525--536 (2013).}

\leavevmode\vadjust pre{\hypertarget{ref-Dickinson2010}{}}%
\CSLLeftMargin{22. }%
\CSLRightInline{Dickinson, J. L., Zuckerberg, B. \& Bonter, D. N. \href{https://doi.org/10.1146/annurev-ecolsys-102209-144636}{Citizen science as an ecological research tool: Challenges and benefits}. \emph{Annual Review of Ecology, Evolution, and Systematics} \textbf{41}, 149--172 (2010).}

\leavevmode\vadjust pre{\hypertarget{ref-Krueger2019}{}}%
\CSLLeftMargin{23. }%
\CSLRightInline{Krueger, K., Esch, L. \& Byrne, R. \href{https://doi.org/10.1371/journal.pone.0218954}{Animal behaviour in a human world: A crowdsourcing study on horses that open door and gate mechanisms}. \emph{PLOS ONE} \textbf{14}, 1--20 (2019).}

\leavevmode\vadjust pre{\hypertarget{ref-Klump2021}{}}%
\CSLLeftMargin{24. }%
\CSLRightInline{Klump, B. C. \emph{et al.} \href{https://doi.org/10.1126/science.abe7808}{Innovation and geographic spread of a complex foraging culture in an urban parrot}. \emph{Science} \textbf{373}, 456--460 (2021).}

\leavevmode\vadjust pre{\hypertarget{ref-Klump2022}{}}%
\CSLLeftMargin{25. }%
\CSLRightInline{Klump, B. C., Major, R. E., Farine, D. R., Martin, J. M. \& Aplin, L. M. Is bin-opening in cockatoos leading to an innovation arms race with humans? \emph{Current Biology} \textbf{32}, R910--R911 (2022).}

\leavevmode\vadjust pre{\hypertarget{ref-Auersperg2009}{}}%
\CSLLeftMargin{26. }%
\CSLRightInline{Auersperg, A. M. I., Gajdon, G. K. \& Huber, L. \href{https://doi.org/10.1098/rsbl.2009.0114}{Kea (\emph{{N}estor notabilis}) consider spatial relationships between objects in the support problem}. \emph{Biology Letters} \textbf{5}, 455--458 (2009).}

\leavevmode\vadjust pre{\hypertarget{ref-Auersperg2010}{}}%
\CSLLeftMargin{27. }%
\CSLRightInline{Auersperg, A. M. I., Gajdon, G. K. \& Huber, L. \href{https://doi.org/10.1016/j.anbehav.2010.08.007}{Kea, \emph{{N}estor notabilis}, produce dynamic relationships between objects in a second-order tool use task}. \emph{Animal Behaviour} \textbf{80}, 783--789 (2010).}

\leavevmode\vadjust pre{\hypertarget{ref-Auersperg2011a}{}}%
\CSLLeftMargin{28. }%
\CSLRightInline{Auersperg, A. M. I., Huber, L. \& Gajdon, G. K. \href{https://doi.org/10.1098/rsbl.2011.0388}{Navigating a tool end in a specific direction: Stick-tool use in kea (\emph{{N}estor notabilis})}. \emph{Biology Letters} \textbf{7}, 825--828 (2011).}

\leavevmode\vadjust pre{\hypertarget{ref-Auersperg2011b}{}}%
\CSLLeftMargin{29. }%
\CSLRightInline{Auersperg, A. M. I., Bayern, A. M. P. von, Gajdon, G. K., Huber, L. \& Kacelnik, A. \href{https://doi.org/10.1371/journal.pone.0020231}{Flexibility in problem solving and tool use of kea and {N}ew {C}aledonian crows in a multi access box paradigm}. \emph{PLOS ONE} \textbf{6}, 1--8 (2011).}

\leavevmode\vadjust pre{\hypertarget{ref-Auersperg2012}{}}%
\CSLLeftMargin{30. }%
\CSLRightInline{Auersperg, A. M. I., Szabo, B., von Bayern, A. M. P. \& Kacelnik, A. \href{https://doi.org/10.1016/j.cub.2012.09.002}{Spontaneous innovation in tool manufacture and use in a {G}offin's cockatoo}. \emph{Current Biology} \textbf{22}, R903--R904 (2012).}

\leavevmode\vadjust pre{\hypertarget{ref-Bastos2021}{}}%
\CSLLeftMargin{31. }%
\CSLRightInline{Bastos, A. P., Horváth, K., Webb, J. L., Wood, P. M. \& Taylor, A. H. Self-care tooling innovation in a disabled kea (\emph{{N}estor notabilis}). \emph{Scientific Reports} \textbf{11}, 18035 (2021).}

\leavevmode\vadjust pre{\hypertarget{ref-BentleyCondit2010}{}}%
\CSLLeftMargin{32. }%
\CSLRightInline{Bentley-Condit, V. \& Smith, E. O. \href{https://doi.org/10.1163/000579509X12512865686555}{Animal tool use: Current definitions and an updated comprehensive catalog}. \emph{Behaviour} \textbf{147}, 185--32A (2010).}

\leavevmode\vadjust pre{\hypertarget{ref-Borsari2005}{}}%
\CSLLeftMargin{33. }%
\CSLRightInline{Borsari, A. \& Ottoni, E. B. \href{https://doi.org/10.1007/s10071-004-0221-3}{Preliminary observations of tool use in captive hyacinth macaws (\emph{{A}nodorhynchus hyacinthinus})}. \emph{Animal Cognition} \textbf{8}, 48--52 (2005).}

\leavevmode\vadjust pre{\hypertarget{ref-Goodman2018}{}}%
\CSLLeftMargin{34. }%
\CSLRightInline{Goodman, M., Hayward, T. \& Hunt, G. R. \href{https://doi.org/10.1038/s41598-018-32363-9}{Habitual tool use innovated by free-living {N}ew {Z}ealand kea}. \emph{Scientific Reports} \textbf{8}, 13935 (2018).}

\leavevmode\vadjust pre{\hypertarget{ref-Heinsohn2017}{}}%
\CSLLeftMargin{35. }%
\CSLRightInline{Heinsohn, R., Zdenek, C. N., Cunningham, R. B., Endler, J. A. \& Langmore, N. E. \href{https://doi.org/10.1126/sciadv.1602399}{Tool-assisted rhythmic drumming in palm cockatoos shares key elements of human instrumental music}. \emph{Science Advances} \textbf{3}, e1602399 (2017).}

\leavevmode\vadjust pre{\hypertarget{ref-Janzen1976}{}}%
\CSLLeftMargin{36. }%
\CSLRightInline{Janzen, M. J., Janzen, D. H. \& Pond, C. M. Tool-using by the {A}frican grey parrot (\emph{{P}sittacus erithacus}). \emph{Biotropica} \textbf{8}, 70 (1976).}

\leavevmode\vadjust pre{\hypertarget{ref-Lambert2015}{}}%
\CSLLeftMargin{37. }%
\CSLRightInline{Lambert, M. L., Seed, A. M. \& Slocombe, K. E. \href{https://doi.org/10.1098/rsbl.2015.0861}{A novel form of spontaneous tool use displayed by several captive greater vasa parrots (\emph{{C}oracopsis vasa})}. \emph{Biology Letters} \textbf{11}, 20150861 (2015).}

\leavevmode\vadjust pre{\hypertarget{ref-OHara2021}{}}%
\CSLLeftMargin{38. }%
\CSLRightInline{O'Hara, M. \emph{et al.} \href{https://doi.org/10.1016/j.cub.2021.08.009}{Wild {G}offin's cockatoos flexibly manufacture and use tool sets}. \emph{Current Biology} \textbf{31}, 4512--4520.e6 (2021).}

\leavevmode\vadjust pre{\hypertarget{ref-Anderson2003}{}}%
\CSLLeftMargin{39. }%
\CSLRightInline{Anderson, P. \href{https://doi.org/10.1163/156853003322796109}{A bird in the house: An anthropological perspective on companion parrots}. \emph{Society \& Animals} \textbf{11}, 393--418 (2003).}

\leavevmode\vadjust pre{\hypertarget{ref-Carrete2008}{}}%
\CSLLeftMargin{40. }%
\CSLRightInline{Carrete, M. \& Tella, J. \href{https://doi.org/10.1890/070075}{Wild-bird trade and exotic invasions: A new link of conservation concern?} \emph{Frontiers in Ecology and the Environment} \textbf{6}, 207--211 (2008).}

\leavevmode\vadjust pre{\hypertarget{ref-Drews2001}{}}%
\CSLLeftMargin{41. }%
\CSLRightInline{Drews, C. \href{https://doi.org/10.1163/156853001753639233}{Wild animals and other pets kept in {C}osta {R}ican households: Incidence, species and numbers}. \emph{Society \& Animals} \textbf{9}, 107--126 (2001).}

\leavevmode\vadjust pre{\hypertarget{ref-Kelly2014}{}}%
\CSLLeftMargin{42. }%
\CSLRightInline{Kelly, D., McCarthy, E., Menzel, K. \& Engebretson, M. \href{https://www.avianwelfare.org/issues/overview.htm}{How many captive birds: Are population studies giving us a clear picture?} (2014).}

\leavevmode\vadjust pre{\hypertarget{ref-Li2014}{}}%
\CSLLeftMargin{43. }%
\CSLRightInline{Li, L. \& Jiang, Z. \href{https://doi.org/10.1371/journal.pone.0085012}{International trade of {CITES} listed bird species in {C}hina}. \emph{PLOS ONE} \textbf{9}, 1--8 (2014).}

\leavevmode\vadjust pre{\hypertarget{ref-Su2015}{}}%
\CSLLeftMargin{44. }%
\CSLRightInline{Su, S., Cassey, P., Vall-llosera, M. \& Blackburn, T. M. \href{https://doi.org/10.1371/journal.pone.0127482}{Going cheap: Determinants of bird price in the {T}aiwanese pet market}. \emph{PLOS ONE} \textbf{10}, 1--17 (2015).}

\leavevmode\vadjust pre{\hypertarget{ref-Young2012}{}}%
\CSLLeftMargin{45. }%
\CSLRightInline{Young, A. M., Hobson, E. A., Lackey, L. B. \& Wright, T. F. \href{https://doi.org/10.1111/j.1469-1795.2011.00477.x}{Survival on the ark: Life-history trends in captive parrots}. \emph{Animal Conservation} \textbf{15}, 28--43 (2012).}

\leavevmode\vadjust pre{\hypertarget{ref-Flammer2001}{}}%
\CSLLeftMargin{46. }%
\CSLRightInline{Flammer, K., Whitt-Smith, D. \& Papich, M. \href{https://doi.org/10.1647/1082-6742(2001)015\%5B0276:PCODIS\%5D2.0.CO;2}{Plasma concentrations of doxycycline in selected psittacine birds when administered in water for potential treatment of \emph{{C}hlamydophila psittaci} infection}. \emph{Journal of Avian Medicine and Surgery} \textbf{15}, 276--282 (2001).}

\leavevmode\vadjust pre{\hypertarget{ref-Iwaniuk2005}{}}%
\CSLLeftMargin{47. }%
\CSLRightInline{Iwaniuk, A. N., Dean, K. M. \& Nelson, J. E. Interspecific allometry of the brain and brain regions in parrots ({P}sittaciformes): Comparisons with other birds and primates. \emph{Brain, Behavior and Evolution} \textbf{65}, 40--59 (2005).}

\leavevmode\vadjust pre{\hypertarget{ref-Mazengenya2018}{}}%
\CSLLeftMargin{48. }%
\CSLRightInline{Mazengenya, P., Bhagwandin, A., Manger, P. R. \& Ihunwo, A. O. \href{https://doi.org/10.3389/fnana.2018.00007}{Putative adult neurogenesis in {O}ld {W}orld parrots: The {C}ongo {A}frican grey parrot (\emph{{P}sittacus erithacus}) and {T}imneh grey parrot (\emph{{P}sittacus timneh})}. \emph{Frontiers in Neuroanatomy} \textbf{12}, 7 (2018).}

\leavevmode\vadjust pre{\hypertarget{ref-Olkowicz2016}{}}%
\CSLLeftMargin{49. }%
\CSLRightInline{Olkowicz, S. \emph{et al.} \href{https://doi.org/10.1073/pnas.1517131113}{Birds have primate-like numbers of neurons in the forebrain}. \emph{Proceedings of the National Academy of Sciences} \textbf{113}, 7255--7260 (2016).}

\leavevmode\vadjust pre{\hypertarget{ref-Schuck2008}{}}%
\CSLLeftMargin{50. }%
\CSLRightInline{Schuck-Paim, C., Alonso, W. J. \& Ottoni, E. B. \href{https://doi.org/10.1159/000119710}{Cognition in an ever-changing world: Climatic variability is associated with brain size in neotropical parrots}. \emph{Brain, Behavior and Evolution} \textbf{71}, 200--215 (2008).}

\leavevmode\vadjust pre{\hypertarget{ref-Silva2017}{}}%
\CSLLeftMargin{51. }%
\CSLRightInline{Silva, T., Guzmán, A., Urantówka, A. D. \& Mackiewicz, P. \href{https://doi.org/10.7717/peerj.3475}{A new parrot taxon from the {Y}ucat{á}n {P}eninsula, {M}exico---its position within genus \emph{{A}mazona} based on morphology and molecular phylogeny}. \emph{PeerJ} \textbf{5}, e3475 (2017).}

\leavevmode\vadjust pre{\hypertarget{ref-Wilman2016}{}}%
\CSLLeftMargin{52. }%
\CSLRightInline{Wilman, H. \emph{et al.} EltonTraits 1.0: Species-level foraging attributes of the world's birds and mammals. (2016) doi:\href{https://doi.org/10.6084/m9.figshare.c.3306933.v1}{10.6084/m9.figshare.c.3306933.v1}.}

\leavevmode\vadjust pre{\hypertarget{ref-Jetz2012}{}}%
\CSLLeftMargin{53. }%
\CSLRightInline{Jetz, W., Thomas, G. H., Joy, J. B., Hartmann, K. \& Mooers, A. O. The global diversity of birds in space and time. \emph{Nature} \textbf{491}, 444--448 (2012).}

\leavevmode\vadjust pre{\hypertarget{ref-Seed2010}{}}%
\CSLLeftMargin{54. }%
\CSLRightInline{Seed, A. \& Byrne, R. \href{https://doi.org/10.1016/j.cub.2010.09.042}{Animal tool-use}. \emph{Current Biology} \textbf{20}, R1032--R1039 (2010).}

\leavevmode\vadjust pre{\hypertarget{ref-Bercovitz1987}{}}%
\CSLLeftMargin{55. }%
\CSLRightInline{Bercovitz, A. B. Avian sex identification techniques. in \emph{Companion bird medicine} (ed. Burr, E. W.) (Iowa State University Press, 1987).}

\leavevmode\vadjust pre{\hypertarget{ref-Hoyo2011}{}}%
\CSLLeftMargin{56. }%
\CSLRightInline{Hoyo, J. D. \& Bierregaard, R. \emph{Handbook of the birds of the world}. (Lynx Edicions, 2011).}

\leavevmode\vadjust pre{\hypertarget{ref-Amico2018}{}}%
\CSLLeftMargin{57. }%
\CSLRightInline{Amico, M. \& Van Keilegom, I. \href{https://doi.org/10.1146/annurev-statistics-031017-100101}{Cure models in survival analysis}. \emph{Annual Review of Statistics and Its Application} \textbf{5}, 311--342 (2018).}

\leavevmode\vadjust pre{\hypertarget{ref-Schmidt1989}{}}%
\CSLLeftMargin{58. }%
\CSLRightInline{Schmidt, P. \& Witte, A. D. \href{https://doi.org/10.1016/0304-4076(89)90034-1}{Predicting criminal recidivism using {`split population'} survival time models}. \emph{Journal of Econometrics} \textbf{40}, 141--159 (1989).}

\leavevmode\vadjust pre{\hypertarget{ref-Wallace1869}{}}%
\CSLLeftMargin{59. }%
\CSLRightInline{Wallace, A. R. \emph{The {M}alay {A}rchipelago: The land of the orang-utan and the bird of paradise; a narrative of travel, with the studies of man and nature}. (MacMillan; Co., 1869).}

\leavevmode\vadjust pre{\hypertarget{ref-Goodall1971}{}}%
\CSLLeftMargin{60. }%
\CSLRightInline{Goodall, J. \href{https://doi.org/10.1016/S0065-3454(08)60157-6}{Tool-using in primates and other vertebrates}. in (eds. Lehrman, D. S., Hinde, R. A. \& Shaw, E.) vol. 3 195--249 (Academic Press, 1971).}

\leavevmode\vadjust pre{\hypertarget{ref-Reader2002}{}}%
\CSLLeftMargin{61. }%
\CSLRightInline{Reader, S. M. \& Laland, K. N. \href{https://doi.org/10.1073/pnas.062041299}{Social intelligence, innovation, and enhanced brain size in primates}. \emph{Proceedings of the National Academy of Sciences} \textbf{99}, 4436--4441 (2002).}

\leavevmode\vadjust pre{\hypertarget{ref-Sol2022}{}}%
\CSLLeftMargin{62. }%
\CSLRightInline{Sol, D. \emph{et al.} Neuron numbers link innovativeness with both absolute and relative brain size in birds. \emph{Nature Ecology \& Evolution} \textbf{6}, 1381--1389 (2022).}

\leavevmode\vadjust pre{\hypertarget{ref-Alvarez2020}{}}%
\CSLLeftMargin{63. }%
\CSLRightInline{Cabrera-Álvarez, M. J. \& Clayton, N. S. \href{https://doi.org/10.3389/fpsyg.2020.560669}{Neural processes underlying tool use in humans, macaques, and corvids}. \emph{Frontiers in Psychology} \textbf{11}, 560669 (2020).}

\leavevmode\vadjust pre{\hypertarget{ref-HenkeVonDerMalsburg2021}{}}%
\CSLLeftMargin{64. }%
\CSLRightInline{Henke-von der Malsburg, J., Kappeler, P. M. \& Fichtel, C. \href{https://doi.org/10.1098/rspb.2021.1728}{Linking cognition to ecology in wild sympatric mouse lemur species}. \emph{Proceedings of the Royal Society B: Biological Sciences} \textbf{288}, 20211728 (2021).}

\leavevmode\vadjust pre{\hypertarget{ref-Fragaszy2018}{}}%
\CSLLeftMargin{65. }%
\CSLRightInline{Fragaszy, D. M. \& Mangalam, M. \href{https://doi.org/10.1016/bs.asb.2018.01.001}{Tooling}. in (eds. Naguib, M. et al.) vol. 50 177--241 (Academic Press, 2018).}

\leavevmode\vadjust pre{\hypertarget{ref-Sandor2020}{}}%
\CSLLeftMargin{66. }%
\CSLRightInline{Sándor, K. \& Miklósi, Á. \href{https://doi.org/10.3389/fpsyg.2020.555487}{How to report anecdotal observations? A new approach based on a lesson from "puffin tool use"}. \emph{Frontiers in Psychology} \textbf{11}, 555487 (2020).}

\leavevmode\vadjust pre{\hypertarget{ref-Debastiani2021}{}}%
\CSLLeftMargin{67. }%
\CSLRightInline{Debastiani, V. J., Bastazini, V. A. G. \& Pillar, V. D. \href{https://doi.org/10.1016/j.ecoinf.2021.101315}{Using phylogenetic information to impute missing functional trait values in ecological databases}. \emph{Ecological Informatics} \textbf{63}, 101315 (2021).}

\leavevmode\vadjust pre{\hypertarget{ref-Hunt2013}{}}%
\CSLLeftMargin{68. }%
\CSLRightInline{Hunt, G. R., Gray, R. D. \& Taylor, A. H. Why is tool use rare in animals? in \emph{Tool use in animals: Cognition and ecology} (eds. Sanz, C. M., Call, J. \& Boesch, C.) 89--118 (Cambridge University Press, 2013). doi:\href{https://doi.org/10.1017/CBO9780511894800.007}{10.1017/CBO9780511894800.007}.}

\leavevmode\vadjust pre{\hypertarget{ref-Benichov2016}{}}%
\CSLLeftMargin{69. }%
\CSLRightInline{Benichov, J. I., Globerson, E. \& Tchernichovski, O. \href{https://doi.org/10.3389/fnhum.2016.00255}{Finding the beat: From socially coordinated vocalizations in songbirds to rhythmic entrainment in humans}. \emph{Frontiers in Human Neuroscience} \textbf{10}, (2016).}

\leavevmode\vadjust pre{\hypertarget{ref-Patel2009}{}}%
\CSLLeftMargin{70. }%
\CSLRightInline{Patel, A. D., Iversen, J. R., Bregman, M. R. \& Schulz, I. \href{https://doi.org/10.1016/j.cub.2009.03.038}{Experimental evidence for synchronization to a musical beat in a nonhuman animal}. \emph{Current Biology} \textbf{19}, 827--830 (2009).}

\leavevmode\vadjust pre{\hypertarget{ref-tenCate2016}{}}%
\CSLLeftMargin{71. }%
\CSLRightInline{ten Cate, C., Spierings, M., Hubert, J. \& Honing, H. \href{https://doi.org/10.3389/fpsyg.2016.00730}{Can birds perceive rhythmic patterns? A review and experiments on a songbird and a parrot species}. \emph{Frontiers in Psychology} \textbf{7}, (2016).}

\leavevmode\vadjust pre{\hypertarget{ref-Wilson2016}{}}%
\CSLLeftMargin{72. }%
\CSLRightInline{Wilson, M. \& Cook, P. F. \href{https://doi.org/10.3758/s13423-016-1013-x}{Rhythmic entrainment: Why humans want to, fireflies can't help it, pet birds try, and sea lions have to be bribed}. \emph{Psychonomic Bulletin \& Review} \textbf{23}, 1647--1659 (2016).}

\leavevmode\vadjust pre{\hypertarget{ref-Eckert2020}{}}%
\CSLLeftMargin{73. }%
\CSLRightInline{Eckert, J., Winkler, S. L. \& Cartmill, E. A. \href{https://doi.org/10.1098/rsbl.2020.0370}{Just kidding: The evolutionary roots of playful teasing}. \emph{Biology Letters} \textbf{16}, 20200370 (2020).}

\leavevmode\vadjust pre{\hypertarget{ref-BroJorgensen2010}{}}%
\CSLLeftMargin{74. }%
\CSLRightInline{Bro-Jørgensen, J. \& Pangle, W. M. \href{https://doi.org/10.1086/653078}{Male topi antelopes alarm snort deceptively to retain females for mating}. \emph{The American Naturalist} \textbf{176}, E33--E39 (2010).}

\leavevmode\vadjust pre{\hypertarget{ref-Byrne1988}{}}%
\CSLLeftMargin{75. }%
\CSLRightInline{Byrne, R. W. \& Whiten, A. \href{https://doi.org/10.1017/S0140525X00049955}{Toward the next generation in data quality: A new survey of primate tactical deception}. \emph{Behavioral and Brain Sciences} \textbf{11}, 267--273 (1988).}

\leavevmode\vadjust pre{\hypertarget{ref-Byrne1991}{}}%
\CSLLeftMargin{76. }%
\CSLRightInline{Byrne, R. W. \& Whiten, A. Computation and mindreading in primate tactical deception. in \emph{Natural theories of mind: Evolution, development and simulation of everyday mindreading} (ed. Whiten, A.) 127--141 (Basil Blackwell, 1991).}

\leavevmode\vadjust pre{\hypertarget{ref-GlobalGfkSurvey}{}}%
\CSLLeftMargin{77. }%
\CSLRightInline{\href{http://www.gfk.com/global-studies/global-studies-pet-ownership}{Global GfK survey: Pet ownership}. (2016).}

\leavevmode\vadjust pre{\hypertarget{ref-Jerison1973}{}}%
\CSLLeftMargin{78. }%
\CSLRightInline{Jerison, H. J. \emph{Evolution of the brain and intelligence}. (Academic Press, 1973).}

\leavevmode\vadjust pre{\hypertarget{ref-Paradis2019}{}}%
\CSLLeftMargin{79. }%
\CSLRightInline{Paradis, E. \& Schliep, K. \href{https://doi.org/10.1093/bioinformatics/bty633}{{ape} 5.0: An environment for modern phylogenetics and evolutionary analyses in {R}}. \emph{Bioinformatics} \textbf{35}, 526--528 (2019).}

\leavevmode\vadjust pre{\hypertarget{ref-McElreath2020}{}}%
\CSLLeftMargin{80. }%
\CSLRightInline{McElreath, R. \emph{\href{http://xcelab.net/rm/statistical-rethinking/}{Statistical rethinking: A {Bayesian} course with examples in {R} and {Stan}, 2nd edition}}. (CRC Press, 2020).}

\leavevmode\vadjust pre{\hypertarget{ref-Stan2020}{}}%
\CSLLeftMargin{81. }%
\CSLRightInline{Stan Development Team. \href{http://mc-stan.org/}{{RStan}: The {R} interface to {Stan}}. (2020).}

\leavevmode\vadjust pre{\hypertarget{ref-phytools}{}}%
\CSLLeftMargin{82. }%
\CSLRightInline{Revell, L. J. \href{https://doi.org/10.1111/j.2041-210X.2011.00169.x}{{phytools}: An {R} package for phylogenetic comparative biology (and other things)}. \emph{Methods in Ecology and Evolution} \textbf{3}, 217--223 (2012).}

\leavevmode\vadjust pre{\hypertarget{ref-RCoreTeam}{}}%
\CSLLeftMargin{83. }%
\CSLRightInline{R Core Team. \emph{\href{https://www.R-project.org/}{R: A language and environment for statistical computing}}. (R Foundation for Statistical Computing, 2022).}

\leavevmode\vadjust pre{\hypertarget{ref-ggtree}{}}%
\CSLLeftMargin{84. }%
\CSLRightInline{Yu, G., Smith, D., Zhu, H., Guan, Y. \& Lam, T. T.-Y. \href{https://doi.org/10.1111/2041-210X.12628}{{ggtree}: An {R} package for visualization and annotation of phylogenetic trees with their covariates and other associated data}. \emph{Methods in Ecology and Evolution} \textbf{8}, 28--36 (2017).}

\leavevmode\vadjust pre{\hypertarget{ref-Wickham2016}{}}%
\CSLLeftMargin{85. }%
\CSLRightInline{Wickham, H. \emph{\href{https://ggplot2.tidyverse.org}{{ggplot2}: Elegant graphics for data analysis}}. (Springer-Verlag New York, 2016).}

\leavevmode\vadjust pre{\hypertarget{ref-Wilke2020}{}}%
\CSLLeftMargin{86. }%
\CSLRightInline{Wilke, C. O. \emph{\href{https://CRAN.R-project.org/package=cowplot}{{cowplot}: Streamlined plot theme and plot annotations for 'ggplot2'}}. (2020).}

\leavevmode\vadjust pre{\hypertarget{ref-Landau2021}{}}%
\CSLLeftMargin{87. }%
\CSLRightInline{Landau, W. M. \href{https://doi.org/10.21105/joss.02959}{The targets {R} package: A dynamic {M}ake-like function-oriented pipeline toolkit for reproducibility and high-performance computing}. \emph{Journal of Open Source Software} \textbf{6}, 2959 (2021).}

\leavevmode\vadjust pre{\hypertarget{ref-Aust2022}{}}%
\CSLLeftMargin{88. }%
\CSLRightInline{Aust, F. \& Barth, M. \emph{\href{https://github.com/crsh/papaja}{{papaja}: {Prepare} reproducible {APA} journal articles with {R Markdown}}}. (2022).}

\end{CSLReferences}

\endgroup

\newpage
\vspace*{60mm}

\renewcommand{\figurename}{Figure}
\renewcommand{\tablename}{Table}
\renewcommand{\thefigure}{S\arabic{figure}} \setcounter{figure}{0}
\renewcommand{\thetable}{S\arabic{table}} \setcounter{table}{0}
\renewcommand{\theequation}{S\arabic{equation}} \setcounter{equation}{0}

\hypertarget{supplementary-information}{%
\section{\texorpdfstring{\textbf{Supplementary Information}}{Supplementary Information}}\label{supplementary-information}}

\setcounter{page}{1}
\centering

\noindent \hspace*{8mm} \small Crowdsourcing and phylogenetic modelling reveal parrot tool use is not rare \newline
\hspace*{1cm} \small Amalia P. M. Bastos\textsuperscript{1,2}, Scott Claessens\textsuperscript{2}, Ximena J. Nelson\textsuperscript{3}, David Welch\textsuperscript{4}, \newline
\hspace*{25mm} Quentin D. Atkinson\textsuperscript{2}, Alex H. Taylor\textsuperscript{2,3,5,6} \newline

\raggedright

\noindent \footnotesize \textsuperscript{1} Department of Psychological \& Brain Sciences, Johns Hopkins University, Baltimore, MD, United States \newline
\noindent \footnotesize \textsuperscript{2} School of Psychology, University of Auckland, Auckland, New Zealand \newline
\noindent \footnotesize \textsuperscript{3} School of Biological Sciences, University of Canterbury, Christchurch, New Zealand \newline
\noindent \footnotesize \textsuperscript{4} School of Computer Science, University of Auckland, Auckland, New Zealand \newline
\noindent \footnotesize \textsuperscript{5} ICREA, Pg. Lluís Companys 23, Barcelona, Spain \newline
\noindent \footnotesize \textsuperscript{6} Institute of Neuroscience, Universitat Autònoma de Barcelona, Barcelona, Spain \newline
\normalsize
\newpage

\hypertarget{supplementary-figures}{%
\subsection{Supplementary Figures}\label{supplementary-figures}}



\begin{figure}
\centering
\includegraphics{manuscript_files/figure-latex/plotSurvCure3-1.pdf}
\caption{\label{fig:plotSurvCure3}\emph{Median posterior probabilities of undetected tool use for each parrot species without observed evidence of tool use from reduced model.} This reduced version of the phylogenetic survival cure model does not contain relative brain size or feeding strategy as predictors, nor does it contain any phylogenetic covariance. The only information included in the model is the number of papers published and the number of YouTube search hits for each species. Each point is a parrot species without observed evidence of tool use, and the colour of the points scales with the probability of undetected tool use. All else being equal, those species with fewer published papers and fewer YouTube search hits have a higher probability of being undetected tool users.}
\end{figure}

\newpage



\begin{figure}
\centering
\includegraphics{manuscript_files/figure-latex/plotSurvCure9-1.pdf}
\caption{\label{fig:plotSurvCure9}\emph{Receiver operating characteristic (ROC) curve for the phylogenetic survival cure model.} The area-under-the-curve in this plot is 0.95, suggesting that the model is able to adequately classify observed tool users and non-tool users.}
\end{figure}

\newpage



\begin{figure}
\centering
\includegraphics{manuscript_files/figure-latex/plotSurvCure5-1.pdf}
\caption{\label{fig:plotSurvCure5}\emph{Prior and posterior phylogenetic covariance functions from the Bayesian survival cure model fitted to the full dataset.} Lines are median posterior functions and shaded areas are 50\% and 95\% credible intervals.}
\end{figure}

\newpage



\begin{figure}
\centering
\includegraphics{manuscript_files/figure-latex/plotSurvCure6-1.pdf}
\caption{\label{fig:plotSurvCure6}\emph{Between-species correlation matrix implied by the posterior phylogenetic covariance function from the Bayesian survival cure model.} Correlations are median posterior estimates. Individual species names omitted for space reasons.}
\end{figure}

\newpage



\begin{figure}
\centering
\includegraphics{manuscript_files/figure-latex/plotComparison-1.pdf}
\caption{\label{fig:plotComparison}\emph{Comparing results between the survival cure model and models fitted to the pre-survey and post-survey data without any survival cure component.} Densities are full posterior distributions from three separate models iterated over 100 posterior parrot phylogenies. Points represent posterior medians, and lines represent 50\% and 95\% credible intervals.}
\end{figure}

\newpage



\begin{figure}
\centering
\includegraphics{manuscript_files/figure-latex/plotASR1-1.pdf}
\caption{\label{fig:plotASR1}\emph{Results of exploratory ancestral state reconstruction analysis fitted to pre-video-survey data, represented on a maximum clade credibility tree.} Tip nodes represent the presence (red) or absence (grey) of observed tool use in the scientific literature. Pie charts represent the posterior probability of tool use presence at each ancestral node.}
\end{figure}

\newpage



\begin{figure}
\centering
\includegraphics{manuscript_files/figure-latex/plotASR2-1.pdf}
\caption{\label{fig:plotASR2}\emph{Results of exploratory ancestral state reconstruction analysis fitted to post-video-survey data, represented on a maximum clade credibility tree.} Tip nodes represent the presence (red) or absence (grey) of observed tool use in the scientific literature and the video survey. Pie charts represent the posterior probability of tool use presence at each ancestral node.}
\end{figure}

\newpage



\begin{figure}
\centering
\includegraphics{manuscript_files/figure-latex/plotASR3-1.pdf}
\caption{\label{fig:plotASR3}\emph{Results of exploratory ancestral state reconstruction analysis fitted to predicted probabilities from the phylogenetic survival cure model, represented on a maximum clade credibility tree.} Tip nodes represent the median posterior predicted probabilities of tool use from the phylogenetic survival cure model, with more red indicating an increasing probability of tool use presence and more grey indicating a decreasing probability of tool use presence. Pie charts represent the posterior probability of tool use presence at each ancestral node.}
\end{figure}

\newpage



\begin{figure}
\centering
\includegraphics{manuscript_files/figure-latex/plotPhylo2-1.pdf}
\caption{\label{fig:plotPhylo2}\emph{Data on encephalisation quotient and feeding strategy for all parrots, presented on a maximum clade credibility tree.} Tip points are coloured according to feeding generalism (orange) and specialism (blue), and scaled according to encephalisation quotient (EQ).}
\end{figure}

\newpage



\begin{figure}
\centering
\includegraphics{manuscript_files/figure-latex/plotPhylo3-1.pdf}
\caption{\label{fig:plotPhylo3}\emph{Data on number of scientific publications until tool use discovery for all parrots, presented on a maximum clade credibility tree.} Tip points are scaled according to the number of published papers up until tool use discovery (or, if tool use has not been observed, the current number of published papers).}
\end{figure}

\newpage



\begin{figure}
\centering
\includegraphics{manuscript_files/figure-latex/plotPhylo4-1.pdf}
\caption{\label{fig:plotPhylo4}\emph{Data on number of video search hits until tool use discovery for all parrots, presented on a maximum clade credibility tree.} Tip points are scaled according to the estimated number of video search hits up until tool use discovery (or, if tool use has not been observed, the current number of video search hits).}
\end{figure}

\newpage



\begin{figure}
\centering
\includegraphics{manuscript_files/figure-latex/plotSurvCure7-1.pdf}
\caption{\label{fig:plotSurvCure7}\emph{Prior predicted probabilities of tool use for each species from our phylogenetic survival cure model.} Points are prior medians and lines are 50\% and 95\% credible intervals.}
\end{figure}

\newpage



\begin{figure}
\centering
\includegraphics{manuscript_files/figure-latex/plotSurvCure8-1.pdf}
\caption{\label{fig:plotSurvCure8}\emph{Results of sensitivity analysis.} The phylogenetic survival cure model was fitted with either a standard Normal(0, 1) prior on the intercept or an alternative Normal(1.78507, 2) prior on the intercept. This latter prior is wider on the logit scale and roughly converts to a 0.86 prior probability of non-tool-use (or a 0.14 prior probability of tool-use, which is the proportion of tool users in the dataset). The sensitivity analysis showed that changing this intercept prior did not have a marked impact on (a) the posterior rankings of parrot species from 1st to 174th or (b) the median posterior probabilities of tool use for parrot species.}
\end{figure}

\newpage



\begin{figure}
\centering
\includegraphics{manuscript_files/figure-latex/plotTrace-1.pdf}
\caption{\label{fig:plotTrace}\emph{Trace plots for the Bayesian phylogenetic survival cure model.} Only four chains are shown for ease of presentation.}
\end{figure}

\newpage



\begin{figure}
\centering
\includegraphics{manuscript_files/figure-latex/plotSurvCureSim-1.pdf}
\caption{\label{fig:plotSurvCureSim}\emph{Posterior estimates from Bayesian survival cure models fitted to 100 datasets simulated with known parameters.} Each dataset consisted of 100 species. Known parameters are presented as solid vertical lines, whereas points and horizontal lines represent posterior medians and 95\% credible intervals. The models were successfully able to recapture the parameters from the simulated datasets.}
\end{figure}


\end{document}
